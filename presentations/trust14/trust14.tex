\documentclass{beamer}

% theme definition
\usetheme{KU}

\usepackage{natbib}
\usepackage{alltt}
\usepackage{trust}

\setbeamertemplate{blocks}[rounded][shadow=true]

\setbeamercolor{title}{fg=kublue}
\setbeamercolor{subtitle}{fg=kugray} 
\setbeamercolor{institute}{fg=kugray}
\setbeamercolor{frametitle}{fg=kublue}
\setbeamercolor{frametitle}{bg=white}
\setbeamercolor{framesubtitle}{fg=kugray}
\setbeamercolor{framesubtitle}{bg=white}
\setbeamercolor{item}{fg=black}
\setbeamercolor{subitem}{fg=kugray}
\setbeamercolor{itemize/enumerate subbody}{fg=kugray}
\setbeamercolor{block title}{bg=kublue}
\setbeamercolor{block title}{fg=white}
\setbeamercolor{block body}{bg=sand}
\setbeamercolor{block body}{fg=black}

\usefonttheme{serif}

\newenvironment{fnverbatim}{\begin{alltt}\scriptsize}{\normalsize\end{alltt}}
\newcommand{\mean}[1]{\langle#1\rangle}
\newcommand{\rtime}{\ensuremath{\mathbb{R}^{0\leq}}}

\bibliographystyle{abbrv}

\title{Trust}
\subtitle{What it is and how to get it}

\author{Dr. Perry Alexander}

%\date{{\color{kugray}\today}}
\date{\ }

% turns off navigation symbols
\setbeamertemplate{navigation symbols}{}

\institute{
    Information and Telecommunication Technology Center \\
    Electrical Engineering and Computer Science \\
    The University of Kansas \\
    \texttt{palexand@ku.edu}}

\begin{document}

\begin{frame}
  \titlepage

{\footnotesize\color{kugray} Formatted with the Beamer Class for \LaTeXe}
\end{frame}

\frame{\frametitle{Defining Trust}
  \begin{block}{Trust}
    ``An entity can be trusted if it always behaves in the exptected
    manner for the intended purpose~\cite{Martin:08:The-ten-page-in}''
  \end{block}
}

\frame{\frametitle{Defining Trust}
  \begin{block}{Properties}
    \begin{itemize}
    \item Unambiguous identification
    \item Unimpeded operation
    \item First-hand observation of good behavior \emph{or} indirect
      experience of good behavior by a trusted third party
    \end{itemize}
  \end{block}
}

\frame{\frametitle{Necessary Capabilities for Trust}
  \begin{itemize}
  \item \emph{Strong Identification} --- An unambiguous, immutable
    identifier associated with the platform.  The identifier is a
    protected encryption key in the TXT implementation.
  \item \emph{Reporting Configuration} --- An unambiguous
    identification mechanism for software and hardware running on the
    platform.  The mechanism is hashing in the TXT implementation
  \end{itemize}
}

\frame{\frametitle{Trust is a Preorder}

  $\trusts{x}{y}$ is an homogeneous relation over actors that is true
  when \emph{x trusts y}.  $\trusts{x}{y}$ is a preorer:

  \begin{itemize}
  \item Reflexive - $\forall x \cdot \trusts{x}{x}$
  \item Transitive - $\forall x,y,z\cdot\trusts{y}{x} \wedge \trusts{z}{y} \Rightarrow
  \trusts{z}{x}$
  \end{itemize}

  The transitive property defines \emph{chains of trust}.
}


\frame{\frametitle{Trusted Platform Module}
  The \emph{Trusted Platform Module (TPM)} is a cryptographic
  coprocessor for trust.

  \begin{itemize}
  \item Endorsement Key (EK) --- factory generated asymmetric key
    that uniquely identifies the TPM
  \item Attestation Instance Key (AIK) --- an alias for the EK
  \item Storage Root Key (SRK) --- user generated asymmetric key that
    encrypts data associated with the TPM
  \item Platform Configuration Registers (PCRs) --- registers for
    storing hashes
  \item NVRAM --- Non-volatile storage associated with the TPM
  \end{itemize}
}

\frame{\frametitle{Platform Configuration Registers}
  \begin{itemize}
  \item Operations on PCRs
    \begin{itemize}
    \item Extension --- Hash a new value juxtaposed with the existing
      PCR value
    \item Reset --- Set to 0
    \item Set --- Set to a known value
    \end{itemize}
  \item Operations using PCRs
    \begin{itemize}
    \item Sealing data --- PCR state dependent encryption
    \item Wrapping keys --- PCR state dependent encryption of a
      private key
    \item Quote --- Reporting PCR values to a third party
    \end{itemize}
  \item Properties
    \begin{itemize}
    \item Locality --- Access control
    \item Resettable --- Can a PCR be reset
    \item Many others that we don't need yet
    \end{itemize}
  \end{itemize}
}

\frame{\frametitle{Roots of Trust} 

  A \emph{root of trust} provides a basis for transitively building
  trust.  Roots of trust are trusted implicitly. 
  \\
  There are three important Roots of Trust:
  
  \begin{itemize}
  \item Root of Trust for Measurement (RTM)
  \item Root of Trust for Reporting (RTR)
  \item Root of Trust for Storage (RTS)
  \end{itemize}
}

\frame{\frametitle{Root of Trust for Measurement}
  A \emph{Root of Trust for Measurement} is trusted to take the base
  system measurement.

  \begin{itemize}
  \item A hash function called on an initial code base from
    a protected execution environment
  \item Starts the measurement process during boot
  \item In the Intel TXT process the RTM is \texttt{SENTER}
    implemented on the processor
  \end{itemize}
}

\frame{\frametitle{Root of Trust for Reporting}
  A \emph{Root of Trust for Reporting} is trusted to authenticate the
  base system report or quote

  \begin{itemize}
  \item A protected key used for authenticating reports
  \item In the Intel TXT processes this is the TPM's Endorsement Key
    (EK)
  \item Created and bound to its platform by the TPM foundry
  \item $\private{EK}$ is stored in the TPM and cannot be accessed by
    any entity other than the TPM
  \item $\public{EK}$ is available for encrypting data for the TPM
  \item $\private{EK}$ is used for decrypting data
  \item Binding of $\public{EK}$ to its platform is maintained by a
    trusted Certificate Authority (CA)
  \end{itemize}
}

\frame{\frametitle{Root of Trust for Storage}
  A \emph{Root of Trust for Storage} is trusted to protect the
  base stored data

  \begin{itemize}
  \item Typically a key stored in a protected location
  \item In the Intel TXT boot process this is the TPM's Storage Root
    Key (SRK)
  \item Created by TPM\_TakeOwnership
  \item $\private{SRK}$ is stored in the TPM
  \item $\private{SRK}$ cannot be accessed directly and can only be
    used by the TPM
  \end{itemize}
}


\frame{\frametitle{One Step from Roots of Trust}
  
  Roots of trust are used to build a trusted system from boot.

  \begin{itemize}
  \item Power On Reset
  \item Resettable PCRs are reset to -1
  \item \texttt{SENTER} resets selected PCRs to 0
    \begin{itemize}
    \item Specified as resettable
    \item PCRs in locality 4 or lower
    \item \emph{Only} \texttt{SENTER} can reset locality 4 PCRs
    \end{itemize}
  \item \texttt{SENTER} hashes \texttt{SINIT} into PCR 18
    \begin{itemize}
    \item RTM generates the first measurement
    \item RTS stores the first measurement in PCR 18
    \end{itemize}
  \end{itemize}
}

\frame{\frametitle{Two Steps from Roots of Trust}

  \begin{itemize}
  \item \texttt{SINIT} measures the Secure Launch Environment (SLE)
    \begin{itemize}
    \item \texttt{SINIT} uses measurement policy stored in the TPM NVRAM
    \item \texttt{SINIT} measured by RTM stored in RTS, thus trusted
    \end{itemize}
  \item \texttt{SINIT} returns control to \texttt{SENTER}
  \item \texttt{SENTER} invokes the SLE
    \begin{itemize}
    \item SLE elements are hashed into PCRs
    \item SLE core measures and starts the operational environment
    \end{itemize}
  \end{itemize}
}

\frame{\frametitle{Chaining Trust}
  \begin{itemize}
  \item Trust is transitive
    \begin{itemize}
    \item $\trusts{x}{y} \wedge \trusts{y}{z} \Rightarrow
      \trusts{x}{z}$
    \item Construct chains of trust
    \item Remember ``directly observed or indirectly observed by a
      trusted third party''
    \end{itemize}
  \item Roots of Trust define the ``root'' for trust
    \begin{itemize}
    \item Use Roots of Trust to establish base for chain
    \item RTM generates a trusted first measurement
    \item RTS protects first measurement
    \item RTR signs base quote for appraiser (eventually)
    \end{itemize}
  \item Extend chains of trust by measuring before executing
  \end{itemize}
}

\frame{\frametitle{Presentation Outline}
  \begin{itemize}
    \item Review access control modeling objectives
      \begin{itemize}
      \item modeling platform MAC
      \item modeling local access control
      \end{itemize}
    \item Overview access control policy definition
      \begin{itemize}
      \item design and modeling assumptions
      \item platform boot policy definition
      \item local policy definitions
      \end{itemize}
    \item Overview models
      \begin{itemize}
      \item domain and system models
      \item communication model
      \item theorems and status
      \end{itemize}
    \item Identify next steps
      \begin{itemize}
      \item runtime and moving beyond the SVP line
      \item adding M\&A detail
      \end{itemize}
  \end{itemize}
}

\frame{\frametitle{Access Control Modeling Objectives}
  \framesubtitle{What we're about here}

  Reporting joint work with Geoffrey Brown, Indiana University (submitted) in which
  we verify two physical layer protocols.
  \begin{itemize}
    \item Biphase Mark Protocol (BMP)
    \item 8N1 Protocol
  \end{itemize}

These protocols are used in data transmission for CDs, Ethernet, and Tokenring,
      etc. as well as UARTs.  
  \begin{itemize}
    \item Correctness is reasonably difficult to prove due to many real-time constraints.

    \item Many previous formal modeling/verification efforts for these protocols.
  \end{itemize}

}

\frame{\frametitle{Columns and Blocks}\framesubtitle{Trying figures
    next to lists}
  \begin{columns}[c]
    \column{.45\textwidth}
%    \begin{block}{Things}
    Some normal text goes here just for introduction
    \begin{itemize}
      \item Appraisal
      \item Measurement
      \item Attestation
      \item vTPM
    \end{itemize}
    \alert{Why is this column getting higher?}\\
    Maybe it's not\\
    Center alignment seems best.\\
    \alert{I like this for two column test and graphics}\\
    Getting higher???
%    \end{block}
    \column{.45\textwidth}
    \begin{figure}
%%      \includegraphics[width=0.95\textwidth]{architecture.pdf}
    \end{figure}
  \end{columns}
}

\frame[plain]{\frametitle{Big Picture}\framesubtitle{Armor Architecture}
  \begin{figure}
%%    \includegraphics[height=0.70\textheight]{architecture.pdf}
  \end{figure}
}
    

{\frame{\frametitle{Simple Block}
  \begin{block}{Introduction to {\LaTeX}}
    ‘‘Beamer is a {\LaTeX}class for creating presentations
    that are held using a projector..."
  \end{block}
  
  \begin{block}{}
    This is a definition
  \end{block}
}

\frame{\frametitle{Proofs}
  \begin{proof}[Not really a proof]
    \begin{enumerate}
    \item<1-3>{This is a step}
    \item<2-3>{This is another step}
    \item<3>{This is a third step}
    \item<3>{This is a third step}
    \item<3>{This is a third step}
    \item<3>{This is a third step}
    \end{enumerate}
  \end{proof}
}


\frame{\frametitle{List with Overlays}
  \begin{itemize}
  \item<1-> Item 1 followed by a pause
  \item<3-> Item 2 followed by a pause
  \item<2-> Item 3 followed by a pause
  \end{itemize}
}

\frame{\frametitle{Previous Efforts}

  \begin{itemize}
  \item BMP has been verified in PVS twice and required
    \begin{itemize}
    \item  37 invariants and 4000 individual proof directives (initially) in the
      one effort
    \item 5 hours just to \emph{check} the proofs in the other effort
    \item A formal specification and verification of an independent real-time model
      in both efforts
    \end{itemize}
  \item BMP has been verified in (the precursor to) ACL2 by J. Moore and required
    \begin{itemize}
     \item A significant conceptual effort to fit the problem in the logic, arguably
       omitting some salient features of the model
     \item The statement and proof of many antecedent results
     \item J. Moore reports this as one of his ``best ideas'' in his career
    \end{itemize}
    
  \end{itemize}
}

\frame{\frametitle{Not Your Father's Theorem-Prover} 

The verifications are carried out in the SAL infinite-state bounded model-checker
that combines SAT-solving and SMT decision procedures to \emph{prove} safety
properties about infinite-state models.

  \begin{itemize}
    \item Theorem-proving efforts took multiple engineer-months if not years to
      complete.

    \item Our initial effort in SAL consumed about \emph{two engineer-days}.\\
      ...and we found a significant bug in a UART application note.
  \end{itemize}
}


\frame[containsverbatim]{\frametitle{Parameterized Timing Constraints}
SMT allows for {\color{red}\emph{parameterized}} proofs of correctness.  The following are
example constaints from the BMP verification:
\begin{fnverbatim}
  TIME: TYPE = REAL;

  TPERIOD: TIME = 16;
  TSAMPLE: INTEGER = 23;
{\color{red}
  TSETTLE}: \{x: TIME |     0 <= x  
                      AND (x + TPERIOD < TSAMPLE) 
                      AND (x + TSAMPLE + 1 < 2 * TPERIOD)\};
{\color{red}
  TSTABLE}: TIME = TPERIOD - TSETTLE;
{\color{red}  
  ERROR}: \{x: TIME |     (0 <= x) 
                    AND (TPERIOD + TSETTLE < TSAMPLE*(1-x)) 
                    AND (TSAMPLE*(1+x) + (1+x) + TSETTLE < 2 * TPERIOD)\};

  RSAMPMAX: TIME = TSAMPLE * (1 + ERROR);
  RSAMPMIN: TIME = TSAMPLE * (1 - ERROR);
  RSCANMAX: TIME = 1 + ERROR;
  RSCANMIN: TIME = 1 - ERROR;
\end{fnverbatim}
}


\frame[containsverbatim]{\frametitle{SRI's SAL Toolset}
      \begin{itemize}
        \item Parser
        \item Simulator
        \item Symbolic model-checker (BDDs) 
        \item Witness symbolic model-checker 
        \item Bounded model-checker
        \item Infinite-state bounded model-checker
        \item Future releases include: 
          \begin{itemize}
            \item Explicit-state model-checker
            \item MDD-based symbolic model-checking
          \end{itemize}
      \end{itemize}
\begin{center}
All of which are ``state-of-the-art''
\end{center}
}


\frame{\frametitle{$k$-Induction}

Please direct your attention to the whiteboard.

}



\frame[label=ta]{ \frametitle{Timeout Automata\footnote{B. Dutertre
  and M. Sorea.  Timed systems in SAL.  \emph{SRI TR}, 2004.} (Semantics)}

  An \emph{explicit} real-time model.

  \begin{itemize}
  \item Vocabulary:
  \begin{itemize}
    \item A set of state variables.
    \item A \emph{global clock}, $c \in \rtime$.
    \item A set of \emph{timeout} variables $T$ such that for $t \in
      T$, $t \in \rtime$.
  \end{itemize}
  \item Construct a transition system $\mean{S, \, S^0, \, \rightarrow}$:
  \begin{itemize}
    \item States are mappings of all variables to values.
    \item Transitions are either \emph{time transitions} or
      \emph{discrete transitions}.
      \begin{itemize}
        \item Time transitions are enabled if the clock is less than
          all timeouts.  Updates clock to least timeout.
        \item Discrete transitions are enabled if the clock equals
          some timeout.  Updates state variables and timeouts.
      \end{itemize}
  \end{itemize}
  \end{itemize}
}

\frame[containsverbatim]{\frametitle{Disjunctive Invariants} Even with $k$-induction, getting a
  sufficiently strong invariant is still hard!  \emph{Disjunctive invariants} help.
  A disjunctive invariant can be built iteratively from the counterexamples returned
  for the hypothesized invariant being verified.

\begin{fnverbatim}
  t0:  THEOREM system |- 
         G(   (    (phase = Settle) 
               AND (rstate = tstate + 1) 
               AND (rclk - tclk - TPERIOD > 0) 
               AND (tclk + TPERIOD + TSTABLE - rclk > 0))
           OR
              (    (phase = Stable) 
               AND (rstate = tstate + 1) 
               AND (rclk - tclk - TSETTLE > 0) 
               AND (tclk + TPERIOD - rclk > 0)  
               AND (rdata = tdata))

                        .
                        .
                        .
\end{fnverbatim}
}

\bibliography{sldg}

\end{document}

