\documentclass[10pt]{article}

%\usepackage{hyperref}
\usepackage{alltt}
\usepackage{natbib}
\usepackage{graphicx}
\usepackage{url}
\usepackage{fancyvrb}
\usepackage{fancyhdr}
\pagestyle{fancy}
\usepackage{trust}
\usepackage{subfigure}
\usepackage{ifthen}
\usepackage{verbatim}

%\usepackage[T1] {fontenc}
%\AtBeginDocument{\catcode`\_=12 \mathcode`\_=\string"715F }
%\DeclareMathAlphabet{\mathtt}{\encodingdefault}{\ttdefault}{m}{n}

%\newcommand{\squash}{}{\itemsep=0pt\parskip=0pt}

\usepackage{tikz}
\usetikzlibrary{arrows,shadows, matrix}
\usepackage[underline=false]{pgf-umlsd}

\lhead{Musings on Protocols and Monads}
\rhead{ArmoredSofware Team}
\lfoot{\copyright The University of Kansas, 2015}
\cfoot{\thepage}

\newboolean{submission}  %%set to true for the submission version
\setboolean{submission}{false}
%\setboolean{submission}{true}
\ifthenelse
{\boolean{submission}}
{ \newcommand{\todo}[1]{ } } % hide todo
{ \newcommand{\todo}[1]{ % show todo
   \marginpar{\raggedright\footnotesize{#1}}
               }}

\newcommand{\squash}{\parskip=0pt\itemsep=0pt}

\newtheorem{assertion}{Assertion}
\newtheorem{definition}{Definition}

\parskip=\medskipamount
\parindent=0pt


\bibliographystyle{abbrvnat}

\title{Musings on Protocols and Monads}
\author{ArmoredSoftware Team \\
 \url{palexand@ku.edu}}

\begin{document}

\maketitle
%\tableofcontents
%\listoffigures
%\listoftables

\begin{abstract}
  This document captures discussions on formally representing
  protocols using monadic constructs.  This is a living document and
  will be updated frequently.
\end{abstract}

\section*{Notation}

Throughout we use a trivial monadic notation for protocols that serves
as our ``assembly language'' target for protocol compilation.
The following conventions hold:

\begin{alltt}
  do \{                % evaluate functions in sequence
       f(x);          % calculate f(x) and discard the result
       y <- f(x);     % calculate f(x) and bind the result to y
       send a \$ x;    % evaluate x and send the result to a
       y <- receive a % receive data from a and the result to y
  \}
\end{alltt}

This is early work, so we play fast and loose with specific syntax and
semantics.  \Verb+send+ and \Verb+receive+ operate synchronously.
Each \Verb+send+ must have a corresponding \Verb+receive+ to complete
its operation.

\section*{Example Protocols}

\subsection*{Needham-Schroeder-Lowe}

\subsubsection*{Message Sequence}
\begin{align*}
& \sndmsg{A}{B} {\encrypt {\nonce{A}, A} {B} } & \\
& \sndmsg{B}{A} {\encrypt {\nonce{A}, \nonce{B}, B} {A}} & \\
& \sndmsg{A}{B} {\encrypt {\nonce{B}} {B}} & \\
\end{align*}

\subsubsection*{Monadic Representation}

\begin{alltt}
  Notes:  
  -Identifiers "a" and "b" serve as principal identity \emph{handles}
  -na, nb, m1, m2, m3 are variables
  -Assume Na and Nb are generated fresh by principals A and B respectively for each run
  -Only A can decrypt using a, and B likewise with b
  -Public keys are known a priori
 
  
  Principal A:
  do \{                
       m1 <- encrypt(\{Na, a\}, b);     % encrypt a's nonce and its i.d. with b's public key
       send b \$ m1;    			              % send result to b
       m2 <- receive b;               % receive (encrypted) message from b
       (na, nb, x) <- decrypt(m2, a); % decrypt m2 using private key of a
       m3 <- encrypt(nb, b);          % encrypt b's nonce with b's public key
       send b \$ m3;                   % send result to b
  \}
\end{alltt}

\begin{alltt}
  Principal B:
  do \{                
       m1 <- receive                  % receive initial (encrypted) message
       (na, a) <- decrypt(m1, b);     % decrypt m1 using b's private key
       m2 <- encrypt(\{na, Nb, b\}, a); % build m2 using na and a
       send a \$ m2;    			              % send result to a
       m3 <- receive a;               % receive (encrypted) message from b
       (nb) <- decrypt(m3, b);        % decrypt it
  \}
\end{alltt}

\subsection*{Wide-Mouthed Frog}
\subsubsection*{Message Sequence}
\begin{align*}
& \sndmsg{A}{S} {\crypt {\nonce{A}, B, \symm{AB}} {\symm{AS}} }& \\
& \sndmsg{S}{B} {\crypt {\nonce{S}, A, \symm{AB}} {\symm{BS}} }& \\
\end{align*}
\subsubsection*{Monadic Representation}


\newpage

\subsection*{CA Protocol}

\subsubsection*{Message Sequence}

%begin constants
\def \pmask {PCR_{m}}
\def \pcomp {PCR_{c}}
\def \evd {d}
\def \eve {e}
\def \cacert {\sign {\public{AIK}}{CA}}
\def \exdata {\hashe{(\eve, \nonce{A}, \cacert ) }}
\def \aikh {AIK_{h}}

\def \app {A}
\def \att {B}
\def \ca {C}
\def \mea {M}
\def \tp {T}

\def \req {R}
\def \resp {P}
\def \k {K}
%end constants

\begin{align*}
  & \sndmsg {\app}{\att} {\evd, \nonce{\app}, \pmask}  & \\
    & \sndmsg{\att}{\tp} {make\_and\_load\_identity} & \\
      & \sndmsg{\tp}{\att} {\aikh} & \\
        & \sndmsg{\att}{\ca} {\att, \public{AIK}}  & \\
  & \sndmsg{\ca}{\att} {\encrypt{K, \hashe{AIK}}{EK}, \encrypt{\cacert}{K}} & \\
    & \sndmsg{\att}{\tp} {activate\_identity({\aikh, \hashe{AIK})}} & \\
  & \sndmsg{\tp}{\att} {K} & \\
    & \sndmsg{\att}{\mea} {\evd}  & \\
  & \sndmsg{\mea}{\att} {\eve}  & \\
  & \sndmsg{\att}{\tp} {quote(\;{AIK_{h}, \pmask, \exdata}\;)} & \\
  & \sndmsg{\tp}{\att} {\pcomp, \sign{\hashe{\pcomp}, \exdata}{AIK}}  & \\
  & \sndmsg{\att}{\app} {\eve, \nonce{\app}, \pcomp, \cacert}  & \\
  & \sndmsg{\att}{\app} {\sign{\hashe{\pcomp}, \exdata}{AIK}}  & 
\end{align*}

\subsubsection*{Strand Space Diagram}

\begin{tikzpicture}[
  implies/.style={double,double equal sign distance,-implies},
  dot/.style={shape=circle,fill=black,minimum size=4pt,
    inner sep=0pt,outer sep=4pt}]

\matrix[matrix of nodes] {
  |[dot,label=above:$\app$] (A1)| {}
  &[4cm] |[dot,label=above:$\att$] (B1)| {}
  &[3.5cm] |[label=above:$\mea$] (M1)| {} 
  &[3.5cm] |[label=above:$\tp$] (T1)| {}
  &[3cm] |[label=above:$\ca$] (C1)| {}\\[1cm]
  & |[dot] (B2)| {}
  &
  & |[dot] (T2)| {}\\[1cm]
  &
  |[dot] (B3)| {}
  &
  & |[dot] (T3)| {}\\  [1cm]
  &
  |[dot] (B4)| {}
  &
  &
  & |[dot] (C4)| {} \\ [1 cm]
  &
  |[dot] (B5)| {}
  &
  &
  & |[dot] (C5)| {} \\ [1 cm]
  &
  |[dot] (B6)| {}
  &
  & |[dot] (T6)| {} \\ [1 cm]
  &
  |[dot] (B7)| {}
  &
  & |[dot] (T7)| {} \\ [1 cm]
  &
  |[dot] (B8)| {}
  & |[dot] (M8)| {} \\ [1 cm]
  &
  |[dot] (B9)| {}
  & |[dot] (M9)| {} \\ [1 cm]
  &
  |[dot] (B10)| {}
  &
  & |[dot] (T10)| {} \\ [1 cm]
  &
  |[dot] (B11)| {}
  &
  & |[dot] (T11)| {} \\ [1 cm]
  |[dot] (A12)| {}
  & |[dot] (B12)| {} \\ [1 cm]
  |[dot] (A13)| {}
  & |[dot] (B13)| {} \\ [1 cm]
};
\draw (A1) edge[->] node[above] {$\evd, \nonce{\app}, \pmask$} (B1);
\draw (B2) edge[->] node[above] {$make\_and\_load\_identity$} (T2)
           edge[implies,implies-] (B1);
\draw (T3) edge[->] node[above] {$\public{AIK}, \aikh$} (B3)
           edge[implies,implies-] (T2);
\draw (B4) edge[->] node[above] {$\public{AIK}$} (C4)
           edge[implies,implies-] (B3);
\draw (C5) edge[->] node[above] {$\encrypt{K, \hashe{\public{AIK}}}{EK}, \crypt{\cacert}{K}$} (B5)
           edge[implies,implies-] (C4);
\draw (B6) edge[->] node[above] {$activate\_identity(\aikh, \encrypt{K, \hashe{\public{AIK}})}{EK})$} (T6)
           edge[implies,implies-] (B5);
\draw (T7) edge[->] node[above] {$K$} (B7)
           edge[implies,implies-] (T6);
\draw (B8) edge[->] node[above] {$\evd$} (M8)
           edge[implies,implies-] (B7);
\draw (M9) edge[->] node[above] {$\eve$} (B9)
           edge[implies,implies-] (M8);
\draw (B10) edge[->] node[above] {$quote(\;\aikh, \pmask, \exdata \;)$} (T10)
% quote(\;{AIK\textsubscript{h}, \pmask, \exdata}\;)
           edge[implies,implies-] (B9);
\draw (T11) edge[->] node[above] {$\pcomp, \sign{\hashe{\pcomp}, \exdata}{AIK}$} (B11)
% \pcomp, sig(\hash{\pcomp}, ed) AIK
           edge[implies,implies-] (T10);
\draw (B12) edge[->] node[above] {$\eve, \nonce{\app}, \pcomp, \cacert$} (A12)
% \eve, \nonce{\app}, \pcomp, \cacert
           edge[implies,implies-] (B11);
\draw (B13) edge[->] node[above] {$\sign{\hashe{\pcomp}, \exdata}{AIK}$} (A13) 
% sig(hash(\pcomp), ed) AIK
           edge[implies,implies-] (B12);
\end{tikzpicture}

\newpage

\subsubsection*{Monadic Representation}

\begin{alltt}
  Appraiser:
  do \{     
       send \att \$ \{ \evd, Na, PCRm \} ;    % 
       \{ e, na, pcrc, cacert, sig \}  <- receive \att
  \}
\end{alltt}

\begin{alltt}
  Attester:
  do \{     
       \{ \evd, na, pcrm \}  <- receive \app;
       send \tp \$ \{ make\_and\_load\_identity \};
       \{ aikpub, aikh \} <- receive \tp;
       send \ca \$ \{ \att, aikpub \};
       \{ ekCipher, kCipher \} <- receive \ca;
       send \tp \$ \{ activate_identity(aikh, ekCipher) \};
       \{ k \} <- receive \tp;
       cacert <- decrypt(kCipher, k);
       send \mea \$ \{ \evd \};
       \{ \eve \} <- receive \tp;
       send \tp \$ \{ quote(aikh, pcrm, \hashe{\eve, na, cacert} \};
       \{ pcrc, sig \} <- receive \tp;
       send \app \$ \{ \eve, na, pcrc, cacert, sig \};
  \}
\end{alltt}

\begin{alltt}
  Measurer:
  do \{     
       \{ \evd \}  <- receive \att;
       \eve <- measure(\evd);
       send \att \$ \{ \eve \};
  \}
\end{alltt}

\begin{alltt}
  CA:
  do \{     
       \{ b, aikpub \}  <- receive \att;
       ekPub <- lookupEkPub(b);
       cacert <- sign(aikpub, c);
       ekCipher <- encrypt( \{ K, \hashe{aikpub} \}, ekPub );
       kCipher <- encrypt( cacert, K);
       send \att \$ \{ ekCipher, kCipher \};
  \}
\end{alltt}


\newpage

\begin{tabular}{l  l  l}
\underline{KEY}  & & \\
\app  & :  &  Appraiser \\
\att  & :  &  Attestation Agent\\
\tp  & :  &  TPM\\
\ca  & :  &  Certificate Authority\\
\mea  & :  &  Measurer\\
d & : & desired evidence \\
e & : & gathered evidence \\
$N_A$ & : & nonce \\
$\pmask$ & : & pcr mask indicating desired pcr registers \\
$\pcomp$  & : & pcr composite structure containing select pcr register values \\
$\aikh$ & : & AIK key handle(used by TPM to reference loaded keys) \\
$\k$ & : & Session key created by \ca \\
\end{tabular}

\bibliography{bundling}

\end{document}
