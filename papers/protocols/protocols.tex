\documentclass[10pt]{article}

%\usepackage{hyperref}
\usepackage{alltt}
\usepackage{natbib}
\usepackage{graphicx}
\usepackage{url}
\usepackage{fancyvrb}
\usepackage{fancyhdr}
\pagestyle{fancy}
\usepackage{trust}
\usepackage{subfigure}
\usepackage{ifthen}

%\newcommand{\squash}{}{\itemsep=0pt\parskip=0pt}

\usepackage{tikz}
\usetikzlibrary{arrows,shadows}
\usepackage[underline=false]{pgf-umlsd}

\lhead{Musings on Protocols and Monads}
\rhead{ArmoredSofware Team}
\lfoot{\copyright The University of Kansas, 2015}
\cfoot{\thepage}

\newboolean{submission}  %%set to true for the submission version
\setboolean{submission}{false}
%\setboolean{submission}{true}
\ifthenelse
{\boolean{submission}}
{ \newcommand{\todo}[1]{ } } % hide todo
{ \newcommand{\todo}[1]{ % show todo
   \marginpar{\raggedright\footnotesize{#1}}
               }}

\newcommand{\squash}{\parskip=0pt\itemsep=0pt}

\newtheorem{assertion}{Assertion}
\newtheorem{definition}{Definition}

\parskip=\medskipamount
\parindent=0pt


\bibliographystyle{abbrvnat}

\title{Musings on Protocols and Monads}
\author{ArmoredSoftware Team \\
 \url{palexand@ku.edu}}

\begin{document}

\maketitle
%\tableofcontents
%\listoffigures
%\listoftables

\begin{abstract}
  This document captures discussions on formally representing
  protocols using monadic constructs.  This is a living document and
  will be updated frequently.
\end{abstract}

\section*{Notation}

Throughout we use a trivial monadic notation for protocols that serves
as our ``assembly language'' target for protocol compilation.
The following conventions hold:

\begin{alltt}
  do \{                % evaluate functions in sequence
       f(x);          % calculate f(x) and discard the result
       y <- f(x);     % calculate f(x) and bind the result to y
       send a \$ x;    % evaluate x and send the result to a
       y <- receive a % receive data from a and the result to y
  \}
\end{alltt}

This is early work, so we play fast and loose with specific syntax and
semantics.  \Verb+send+ and \Verb+receive+ operate synchronously.
Each \Verb+send+ must have a corresponding \Verb+receive+ to complete
its operation.

\bibliography{bundling}

\end{document}
