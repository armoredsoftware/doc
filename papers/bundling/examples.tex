\documentclass%[12pt]
{llncs}  % report article

\newif\ifieee
\ieeefalse		%\ieeetrue

\newif\ifllncs
\llncstrue		%\llncsfalse

\newif\ifamsart
\amsartfalse             %\amsarttrue


\newif\ifproofs
\proofstrue             %\proofsfalse

%\usepackage{times}
\normalfont
\usepackage[T1]{fontenc}


%\ifllncs\usepackage{smallsubsub}\fi
%\input{macros} 


\usepackage{enumerate}
\usepackage[all]{xypic}
\usepackage{multirow}
\usepackage{csquotes}
%\usepackage{hyperref}


%macros
\newcommand{\enc}[2]{\{\!|#1|\!\}_{#2}}



\pagestyle{plain}

\title{Examples of Multi-Realm Attestation}
\author{Paul D. Rowe \and Joshua D. Guttman}
\institute{The MITRE Corporation}
\date{\today}

\begin{document}
\maketitle


\section*{A Simple Protocol}
We start from the same example explored by Perry, namely the problem
of determining the status of a virus checker on a remote system. We
assume the appraisal happens as part of the entrance to a wireless
network. The appraiser wants to ensure that, at the time of entrance
to the network, the target system has a virus checker that is the
current version and is using the expected signature files. The
appraiser is willing to live with the risk that a compromise happens
once the target is on the network. 

We assume the appraiser wants to see the virus checker version number
$vn$ and (a hash of?) the signature file $sf$ being used. (We ignore
the source of the signature file for now.) The following depicts a
possible expected message flow of a protocol consistent with Perry's
description. We instantiate some choices that will turn out to be
problematic to demonstrate the kinds of things that can go wrong. 

\begin{figure}[h]
  \centering
  \begin{displaymath}
  \xymatrix{
  APB & & ASP & & TPM \\
  \bullet \ar[rr]^{n,mspec}
         \ar@{=>}[dddd] & & \bullet \ar@{=>}[d] & & ~\\
   ~& & \circ \ar@{}[r]|{~~~~~meas:~vn,~sf}\ar@2{->}[d] & & ~\\
   ~& & \bullet \ar@2{->}[d]\ar[rr]^{quote~req.~n,~vn,~sf} & & \bullet
   \ar@2{->}[d]\\
   ~& & \bullet \ar@2{->}[d] & & \bullet \ar[ll]_{quote}\\
   \bullet & & \bullet \ar[ll]_{bundle} & &~}
  \end{displaymath}
 \begin{tabular}[h]{rl}
   $quote:$ & $\enc{h(n,vn,sf),PCRComp_0}{AIK_0}$\\
   $bundle:$ & $\langle (vn,sf), quote\rangle$
 \end{tabular}
  \caption{Message flow for simple protocol}
  \label{fig:protocol flow}
\end{figure}

The APB forwards to the relevant ASP the measurement spec (that
specifies the version number and signature file) and a nonce $n$
provided by the appraiser. The ASP takes the measurements, and then
requests a quote from the TPM (possibly vTPM). We might assume that
$PCRComp_0$ reflects the load-time state of the M\&A domain, or at
least the APB and ASP. The ASP may then bundle the quote together
with the plaintext of the measurements and hand it back to the APB.

In this example, $vn$ and $sf$ represent the primary evidence, while
the bundle is intended to provided secondary evidence about how the
primary evidence was collected. We are particularly interested in how
an appraiser might infer what happened on the system in order to
determine that its desired goal is met. Let's assume for now that the
appraiser is willing to trust the APB and ASP on the basis of the
load-time information conveyed in $PCRComp_0$, and that this trust is
well-placed. That is, suppose those components have not been
compromised. Then the appraiser would like to trust the quote because
it came from a legitimate TPM and the trustworthy ASP made the
request. Because the ASP is trustworthy, it will only include the true
measurements it took. 

\subsection*{An attack}

It turns out the above protocol will not ensure the target has
up-to-date signature files and version number. To see this, consider
what an attacker might do. At any time prior to the attestation
request, the attacker can change the signature file to ensure it
cannot be detected by the virus checker. When it detects the
attestation request, it can grab the appraiser's nonce and directly
request a quote from the TPM using the expected values instead of the
actual values. The attacker might then ensure this quote is sent back
to the appraiser instead of the quote requested by the ASP that would
contain the actual (and unacceptable) measurement values. The end
result is that the appraiser is fooled into thinking the ASP took
measurements that indicated acceptable values, when in fact, the ASP
measurement yielded unacceptable values.

The problem is obviously not with the measurement gathering process
since that part is not tampered with. Rather, the problem is that the
quote is insufficient to allow the appraiser to infer who provided the
measurement values to the TPM. This particular bundling mechanism is
consistent with several executions of the platform, one of which
defeats the goals of the attestation. This is our first example of a
bundling mechanism being untrustworthy. It also suggests a first rule
of thumb for bundling mechanisms:

\begin{displayquote}
  \textbf{Rule of Thumb~1:} Evidence should not be incorporated into a
  quote solely in the ExternalData (a.k.a. nonce) field. A TPM quote
  should always indicate which component provided the evidence to the
  TPM.
\end{displayquote}

\subsection*{A mitigation}
The above rule of thumb suggests that we should modify the protocol to
ensure the appraiser can correctly determine that the source of the
measurement values is the ASP. Below is an initial suggestion.


\begin{figure}[h]
  \centering
  \begin{displaymath}
  \xymatrix{
  APB && ASP && TPM \\
  \bullet \ar[rr]^{n,mspec}
         \ar@{=>}[ddddd] && \bullet \ar@{=>}[d] &&\\
   && \circ \ar@{}[r]|{~~~~~meas:~vn,~sf}\ar@2{->}[d] &&\\
   && \bullet \ar@2{->}[d]\ar[rr]^{extend~vn,~sf} &&\bullet \ar@2{->}[d]\\
   && \bullet \ar@2{->}[d]\ar[rr]^{quote~req.~n} && \bullet
   \ar@2{->}[d]\\
   && \bullet \ar@2{->}[d] && \bullet \ar[ll]_{quote}\\
   \bullet && \bullet \ar[ll]_{bundle} &&}
  \end{displaymath}
 \begin{tabular}[h]{rl}
   $quote:$ & $\enc{n,PCRComp_1}{AIK_0}$\\
   $bundle:$ & $\langle (vn,sf), quote\rangle$
 \end{tabular}
  \caption{Message flow for amended protocol}
  \label{fig:protocol amended}
\end{figure}

This protocol has the ASP extending the measurement values into a TPM
PCR, which could be a resettable PCR to which the ASP has exclusive
access. The quote now includes a new PCR composite value $PCRComp_1$
that covers PCRs reflecting the load-time state of the ASP and also
the newly included PCR that contains info about the status of the
virus checker. 

Although this protocol provides the same evidence to the appraiser
(namely, $vn$ and $sf$) as the original protocol, the bundling
mechanism allows for fewer executions on the target platform that are
consistent with the generation of the bundle. In particular, a
corrupted user machine can no longer create a quote with the expected
measurement values in the right location. If the ASP is trustworthy,
and if the ASP truly has exclusive access to the resettable PCR, then
the appraiser can believe the values reported in the quote are the
result of the ASP's measurement.

\subsection*{Towards multi-realm attestations}
So far, we have used a simple, single-realm attestation protocol to
demonstrate the way in which one bundling mechanism may be preferred
to another that bundles the same primary evidence. The improved
bundling mechanism prevents undesirable executions that the original
mechnaism allows, but only under explicit trust assumptions about the
lower layers of the system. This suggests that the appraiser would
like to recieve more evidence about the underlying system in order to
validate those trust assumptions. Of course, depending on the context,
the target system may be unwilling to provide such further
evidence. This is the basic tension between the appraiser's policy and
the target's policy. 

Extending the particular example from above, the appraiser needs
evidence that the ASP is trustworthy at the time of the attestation,
and that the relevant combination of policies are in place to ensure
that the ASP has exclusive access to the relevant PCR. Depending on
the appraiser's policy, it may be willing to settle for the load-time
measurements of the ASP and relevant policies, or it might require a
runtime re-measurement of both. In the former case, the amended
protocol of Figure~\ref{fig:protocol amended} would be sufficient. 

If, however, the appraiser believes the ASP could be corrupted after
load-time, or the relevant policy could be changed, then that protocol
is insufficient. If the ASP can be corrupted at runtime, then the
load-time measurement of it, reflected in $PCRComp_1$, will not detect
the corruption. This corrupted ASP could use its privilege to extend
the expected measurement values into the restricted PCR even if the
signature file has been modified. Similarly, if the policy is modified
to allow unrestricted access to the PCR in question, then the attack
described above---in which an attacker on the user VM creates a valid
quote with expected values---is again possible. 

As this note evolves, we will want to explore some strategies for
collecting the lower-level measurements of the ASP and access control
policies and bundling it with the evidence of the virus checker. There
are many more options both for how to gather the evidence (it may
matter whether the lower-level measurements are taken first) and how
to bundle the evidence from the various realms. Here is a brief list
of the types of things we expect to find with examples of poorly
chosen mechanisms:

\begin{itemize}
  \item Attacker can reuse old measurements of the low-level
    components
  \item Attacker can request fresh, clean measurements of another
    machine
  \item A compromised ASP might be able to forge lower-level quotes
    with its TPM using a different AIK
  \item A compromised user-level attestation manager may negotiate a
    different attestation in bad faith allowing it to evade detection
\end{itemize}

The last item begins to hint at the issues surrounding policy
distribution and enforcement that we have discussed before. We will
want to explore the consequences of different choices for how to
negotiate attestation protocols. 







\end{document}
