\documentclass[10pt]{article}

%\usepackage{hyperref}
\usepackage{alltt}
\usepackage{natbib}
\usepackage{graphicx}
\usepackage{url}
\usepackage{fancyvrb}
\usepackage{fancyhdr}
\pagestyle{fancy}
\usepackage{trust}
\usepackage{subfigure}
\usepackage{ifthen}

\usepackage{tikz}
\usetikzlibrary{arrows,shadows}
\usepackage[underline=false]{pgf-umlsd}

\lhead{ArmoredSoftware Architecture}
\rhead{Alexander, Gill, Kulkarni, Searl}
\lfoot{\copyright The University of Kansas, 2013}
\cfoot{\thepage}


\newboolean{submission}  %%set to true for the submission version
\setboolean{submission}{false}
%\setboolean{submission}{true}
\ifthenelse
{\boolean{submission}}
{ \newcommand{\todo}[1]{ } } % hide todo
{ \newcommand{\todo}[1]{ % show todo
   \marginpar{\raggedright\footnotesize{#1}}
               }}
\newcommand{\squash}{\parskip=0pt\itemsep=0pt}

\parskip=\medskipamount
\parindent=0pt


\bibliographystyle{abbrvnat}

\title{Musings on Gathering and Bundling Evidence}
\author{Perry Alexander \\
 \url{palexand@ku.edu}}

\begin{document}

\maketitle
%\tableofcontents
%\listoffigures
%\listoftables

\begin{abstract}
  This document captures discussions on evidence bundling in semantic
  remote attestation.  Evidence Bundling is described as aggregation
  of evidence from multiple sources into a trustworthy package.  This
  necessarily includes primary evidence from the system being
  appraised and meta-evidence describing the evidence gathering
  process.  This is a living document and will be updated frequently.
\end{abstract}

\section{Introduction}

Consider the problem of determining the status of virus checking on a
remote system.  This might occur when a new computer is asking to join
a controlled network such as a university wireless system.  The
appraiser representing the university wireless system wants to
determine the status of an attestation manager representing the new
computing system.

On the appraiser side a trivial protocol runs sending a request
(\Verb+r+) to a target (\Verb+t+) and in response receiving a protocol
(\Verb+Some q+) or refusal to participate (\Verb+None+):

\begin{alltt}
  do \{ sndRequest(r,t);
       q <- rcvProt(t)
       e <- case q of
              Some p : sndRemote(p,t)
              None : None
            end;
       case e of 
         Some v : appraise(v)
         None : None
       end
  \}
\end{alltt}

If the request returns a protocol (\Verb+p+), the appraiser sends the
protocol to execute on the target. Evidence (\Verb+e+) is received
from the target and appraised if it is valid evidence.

On the attestation manager, a protocol runs that receives a request
from the appraiser (\Verb+a+), evaluates its privacy policy
(\Verb+priv+) with respect to the request and the appraiser's ID.  If
its privacy policy is respected and the appraiser trusted, it returns
a single attestation protocol (\Verb+p0+).  Otherwise, it refuses to
participate (\Verb+None+):

\begin{alltt}
  do \{ (r,a) <- rcvRequest;
       p <- if (priv r a) then Some p0 else None;
       sndProt(p,a);
       (q,a) <- rcvRemote;
       e <- if p=q then execute(q) else None;
       sndEvidence(e,a)
  \}
\end{alltt}

There is no negotiation, just a simple determination if the request
satisfies the target's privacy policy.  If so, the only protocol that
can be returned and executed is \Verb+p0+.

The interface between this negotiation protocol and the attestation
protocol resulting from it is the \Verb+execute+ function that causes
the attestation protocol to run on the attestation manager.

To consider bundling, we look at several examples of an attestation
protocols (\Verb+p0+) for determining the status of a target's virus
checking software.

First is a protocol that assesses several properties and runs locally
on the attestation manager's system:

\begin{alltt}
  do \{ id <- getVCID;
       sig <- getSigFileEvidence;
       src <- getSigFileSrc;
       e <- createEvidence(id,sig,src);
       returnEvidence(e)
  \}
\end{alltt}

This protocol gets the checker ID (\Verb+getVCID+), checks the
signature file (\Verb+getSigFileEvidence+), and checks the source of
the signature file (\Verb+getSigFileSrc+). Then it bundles all
evidence into a single evidence package (\Verb+createEvidence+) and
returns it (\Verb+returnEvidence+) to be sent back to the appraiser by
the negotiation protocol.  The appraiser then consumes the resulting
evidence to perform appraisal (\Verb+appraise+).

The evidence returned has approximately the following form:

\[\evidence{(id,sig,src)}{\hashe{(id,sig,src)},PCRComp_0}{AIK_0}\]

where $(id,sig,src)$ is primary evidence and hashes and signatures are
meta-evidence.  An appraiser can check: (i) primary evidence to assess
the measured virus checking subsystem; and (ii) the signature on the
quote to determine its authenticity and $PCRComp$ to assess the
platform construction.  This is a trivial example of what we call
bundling---combining primary evidence and meta-evidence in the same
evidence package.

Another protocol represents simply telling the appraiser a virus
checker is running:

\begin{alltt}
  do \{ id <- getVCID;
       e <- createEvidence(id);
       returnEvidence(e)
  \}
\end{alltt}

This simpler protocol generates less evidence than the first, but may
be more acceptable to the target system.  The appraiser could
determine this is insufficient or limit access to resources and
services based on its appraisal.

%% Negotiation example here

A more interesting case for the attestation protocol happens in the
presence of more complex bundling:

\begin{alltt}
  do \{ id <- getVCID;
       sig <- getSigFileEvidence;
       src <- getSigFileSrc;
       srcQuote <- appraiseSrc(src);
       e <- createEvidence(id,sig,src,srcEvidence)
       returnEvidence(e)
  \}
\end{alltt}

In this case the \Verb+appraiseSrc+ function does a full appraisal of
the signature file server, \Verb+src+. The function communicates its
need for evidence from the signature file source to the appraiser
associated with the attestation manager.  That attestation manager
then executes a similar negotiation protocol as the one that started
the appraisal process.  The resulting evidence, \Verb+srcEvidence+,
has the form:

\[\evidence{(e)}{\hashe{e},PCRComp_1}{AIK_1}\]

is signed by the signature file server, not the original appraisal
target with its own AIK. Thus, the information from the encapsulated
quote is bundled in the outer quote:

\[b=\evidence{(e)}{\hashe{e},PCRComp_1}{AIK_1}\]
\[\evidence{(id,sig,src,b)}{\hashe{(id,sig,src,b)},PCRComp_0}{AIK_0}\]

The appraiser must somehow interpret the quote from the signature file
server without \emph{a priori} knowledge of its identity.

This more general version of the bundling problem requires:

\begin{itemize}
  \parskip=0pt\itemsep=0pt
\item Strong identification of the signature file server to the appraiser
\item Nested evaluation of the privacy policy for the ``other''
  attestation manager before executing the entire attestation
  protocol.
\end{itemize}

An alternative implementation would have the attestation manager
return the identity of the signature server and allow the appraiser to
negotiate separately with the signature server.  This could be called
\emph{flattening} the protocol if there is a desire to name it.

\begin{alltt}
  do \{ sndRequest(r0,t0);
       q0 <- rcvProt(t0);
       e0 <- case q0 of
               Some p : sndRemote(p,t0)
               None : None
             end;
       r1 <- appraise(e0);
       
       sndRequest(r1,t1);
       q1 <- rcvProt(t);
       e1 <- case q1 of
               Some p : sndRemote(p,t1)
               None : None
             end;
       appraise(e1)
  \}
\end{alltt}

Is this flattening operation---eliminating hierarchy in the
attestation protocol---a general operation or something that applies
only here?  If it is not general, can we live with its limitations?
Can we assert any correctness properties for it?  Could there be an
interesting man-in-the-middle attack where the outer attestation
manager could negotiate in bad faith and return instructions to the
appraiser that produce a bad result?

\appendix

\section{Glossary}

\begin{description}
\item[Primary Evidence] Evidence describing the measured application.
\item[Meta-Evidence] Evidence describing properties of other evidence.
\item[Negotiation Protocol] Sequence of events used to determine what
  protocol(s) to run and when to run it/them.
\item[Attestation Protocol] Sequence of events used to gather and
  prepare evidence by invoking attestation service providers.
\item[Attestation Protocol Block (APB)] See Attestation Protocol
  Instance (API).
\end{description}

%%\nocite{}

\bibliography{bundling}

\end{document}
