\documentclass[10pt]{article}

%\usepackage{hyperref}
\usepackage{alltt}
\usepackage{natbib}
\usepackage{graphicx}
\usepackage{url}
\usepackage{fancyvrb}
\usepackage{fancyhdr}
\pagestyle{fancy}
\usepackage{trust}
\usepackage{subfigure}
\usepackage{ifthen}

%\newcommand{\squash}{}{\itemsep=0pt\parskip=0pt}

\usepackage{tikz}
\usetikzlibrary{arrows,shadows}
\usepackage[underline=false]{pgf-umlsd}

\lhead{Gathering and Bundling Evidence}
\rhead{Alexander}
\lfoot{\copyright The University of Kansas, 2015}
\cfoot{\thepage}


\newboolean{submission}  %%set to true for the submission version
\setboolean{submission}{false}
%\setboolean{submission}{true}
\ifthenelse
{\boolean{submission}}
{ \newcommand{\todo}[1]{ } } % hide todo
{ \newcommand{\todo}[1]{ % show todo
   \marginpar{\raggedright\footnotesize{#1}}
               }}

\newcommand{\squash}{\parskip=0pt\itemsep=0pt}

\newtheorem{assertion}{Assertion}

\parskip=\medskipamount
\parindent=0pt


\bibliographystyle{abbrvnat}

\title{Musings on Gathering and Bundling Evidence}
\author{Perry Alexander \\
 \url{palexand@ku.edu}}

\begin{document}

\maketitle
%\tableofcontents
%\listoffigures
%\listoftables

\begin{abstract}
  This document captures discussions on evidence bundling in semantic
  remote attestation.  Evidence Bundling is described as aggregation
  of evidence from multiple sources into a package whose overall
  trustworthiness can be appraised.  This necessarily includes primary
  evidence from the system being appraised and meta-evidence
  describing the evidence gathering process.  This is a living
  document and will be updated frequently.
\end{abstract}

\section*{Notation}

Throughout we use a trivial monadic notation for protocols that serves
as our ``assembly language'' target for protocol compilation.
The following conventions hold:

\begin{alltt}
  do \{                % evaluate functions in sequence
       f(x);          % calculate f(x) and discard the result
       y <- f(x);      % calculate f(x) and bind the result to y
       sndXXX(x,a);   % operation on x that sends something to a
       y <- rcvXXX(a) % an operation that receives something from a
                      %  and binds it to y
  \}
\end{alltt}

This is early work, so we play fast and loose with specific syntax and
semantics.

\section*{Bundling}

A working definition of bundling from ongoing discussions:

\begin{quotation}
  ``If an attestation [protocol] is negotiated and an appraiser receives
  evidence that some set of components is uncompromised, then
  ensure the appraiser can verify that those uncompromised
  components executed their part of the plan'' -- Paul Rowe and Joshua
  Guttman
\end{quotation}

Said differently, bundling is the assembly of evidence from multiple
sources in a way that enables an appraiser to determine that each
element was obtained by the correct measurement component at the
correct time.

Following we will outline working examples of when and how bundling
occurs.  We will start by defining a simple negotiation among an
appraiser and attestation manager.  Then we will define several
potential results of that negotiation and discuss how those examples
reflect bundling issues.

\subsection*{Negotiation}

Consider the problem of determining the status of virus checking on a
remote system.  This might occur when a new computer is asking to join
a controlled network such as a university wireless system.  The
appraiser representing the university wireless system wants to
determine the status of an attestation manager representing the new
computing system.

On the appraiser side a trivial protocol runs sending a request
(\Verb+r+) to a target (\Verb+t+) and in response receiving a protocol
(\Verb+Some p+) or refusal to participate (\Verb+None+):

\begin{alltt}
  do \{ sndRequest(r,t);
       q <- rcvProt(t)
       e <- case q of
              Some p : sndRemote(p,t)
              None : None
            end;
       case e of 
         Some v : appraise(v)
         None : None
       end
  \}
\end{alltt}

If the request returns a protocol (\Verb+p+), the appraiser sends the
protocol to execute on the target. Evidence (\Verb+e+) is received
from the target and appraised if it is valid evidence.

On the attestation manager, a protocol runs that receives a request
from the appraiser (\Verb+a+), evaluates its privacy policy
(\Verb+priv+) with respect to the request and the appraiser's ID.  If
its privacy policy is respected and the appraiser trusted, it returns
a single attestation protocol (\Verb+p0+).  Otherwise, it refuses to
participate (\Verb+None+):

\begin{alltt}
  do \{ (r,a) <- rcvRequest;
       p <- if (priv r a) then (Some p0) else None;
       sndProt(p,a);
       (q,a) <- rcvRemote;
       e <- if p=q then execute(q) else None;
       sndEvidence(e,a)
  \}
\end{alltt}

There is no negotiation, just a simple determination if the request
satisfies the target's privacy policy.  If so, the only protocol that
can be returned and executed is \Verb+p0+.

The interface between this negotiation protocol and the attestation
protocol resulting from it is the \Verb+execute+ function that causes
the attestation protocol to run on the attestation manager.

\subsection*{Attestation}

To consider bundling, we look at several examples of an attestation
protocols (\Verb+p0+) for determining the status of a target's virus
checking software.

First is a protocol that assesses several properties and runs locally
on the attestation manager's system:

\begin{alltt}
  do \{ id <- getVCID;
       sig <- getSigFileEvidence;
       src <- getSigFileSrc;
       e <- createEvidence(id,sig,src);
       returnEvidence(e)
  \}
\end{alltt}

This protocol gets the checker ID (\Verb+getVCID+), checks the
signature file (\Verb+getSigFileEvidence+), and checks the source of
the signature file (\Verb+getSigFileSrc+). Then it bundles all
evidence into a single evidence package (\Verb+createEvidence+) and
returns it (\Verb+returnEvidence+) to be sent back to the appraiser by
the negotiation protocol.  The appraiser then consumes the resulting
evidence to perform appraisal (\Verb+appraise+).

The evidence returned has approximately the following form:

\[\evidence{(id,sig,src)}{\hashe{(id,sig,src)},PCRComp_0}{AIK_0}\]

where $(id,sig,src)$ is primary evidence and hashes and signatures are
meta-evidence.  An appraiser can check: (i) primary evidence to assess
the measured virus checking subsystem; and (ii) the signature on the
quote to determine its authenticity and $PCRComp$ to assess the
platform construction.

This is a simple example of evidence bundling---combining primary
evidence from three ASPs and meta-evidence from the appraisal
environment in the same package.  In this case, primary evidence gives
us information about the running system and meta-evidence gives us
information about how the evidence was gathered.  The appraiser first
needs to determine if the quote and $PCRComp$ are sufficient
meta-evidence for it to trust its primary evidence and then determine
if $(id,sig,src)$ is sufficient to trust that the remote system is
running an appropriately configured virus checker.

Another protocol represents simply providing the appraiser evidence
that a virus checker is running:

\begin{alltt}
  do \{ id <- getVCID;
       e <- createEvidence(id);
       returnEvidence(e)
  \}
\end{alltt}

This simpler protocol generates less evidence than the first, but may
be more acceptable to the target system.  The appraiser could
determine this is insufficient or limit access to resources and
services based on its appraisal.

\emph{Note: This example will expand into a more involved negotiation
  of a protocol.}

A more interesting case for the attestation protocol happens in the
presence of complex bundling:

\begin{alltt}
  do \{ id <- getVCID;
       sig <- getSigFileEvidence;
       src <- getSigFileSrc;
       srcEvidence <- appraiseSrc(src);
       e <- createEvidence(id,sig,src,srcEvidence)
       returnEvidence(e)
  \}
\end{alltt}

In this case the \Verb+appraiseSrc+ function does a full appraisal of
the signature file server, \Verb+src+. The function communicates its
need for evidence from the signature file source to the appraiser
associated with the attestation manager.  That attestation manager
then executes a similar negotiation protocol as the one that started
the appraisal process.  The resulting evidence, \Verb+srcEvidence+,
has the form:

\[\evidence{(e)}{\hashe{e},PCRComp_1}{AIK_1}\]

is signed by the signature file server, not the original appraisal
target with its own AIK. Thus, the information from the encapsulated
evidence is bundled in the outer evidence:

\[b=\evidence{(e)}{\hashe{e},PCRComp_1}{AIK_1}\]
\[\evidence{(id,sig,src,b)}{\hashe{(id,sig,src,b)},PCRComp_0}{AIK_0}\]

The appraiser must somehow interpret the evidence from the signature
file server without \emph{a priori} knowledge of its identity.

This more general version of the bundling problem requires:

\begin{itemize}
  \parskip=0pt\itemsep=0pt
\item Strong identification of the signature file server to the appraiser
\item Nested evaluation of the privacy policy for the ``other''
  attestation manager before executing the entire attestation
  protocol.
\end{itemize}

An alternative implementation would have the attestation manager
return the identity of the signature server and allow the appraiser to
negotiate separately with the signature server.  This could be called
\emph{flattening} the protocol if there is a desire to name it.

\begin{alltt}
  do \{ sndRequest(r0,t0);
       q0 <- rcvProt(t0);
       e0 <- case q0 of
               Some p : sndRemote(p,t0)
               None : None
             end;
       r1 <- appraise(e0);
       
       sndRequest(r1,t1);
       q1 <- rcvProt(t);
       e1 <- case q1 of
               Some p : sndRemote(p,t1)
               None : None
             end;
       appraise(e1)
  \}
\end{alltt}

The evidence produced from both requests from the appraiser has the
following form:

\[\evidence{(e)}{\hashe{e},PCRComp_1}{AIK_1}\]
\[\evidence{(id,sig,src)}{\hashe{(id,sig,src)},PCRComp_0}{AIK_0}\]

Note that this is quite similar to the evidence produced by the
earlier, un-flattened attestation.  We have the same PCR composites
and evidence signed by the same AIK instances.  The distinction is the
evidence packages are not nested.  Semantically, what is the
distinction between these two results?  Is one preferable to another
from an evidence perspective?

Is this flattening operation---eliminating hierarchy in the
attestation protocol---a general operation or something that applies
only here?  If it is not general, can we live with its limitations?
Can we assert any correctness properties for it?  Could there be an
interesting man-in-the-middle attack where the outer attestation
manager could negotiate in bad faith and return instructions to the
appraiser that produce a bad result?

\section*{Privacy Policies}

Each protocol executed in the virus checking example is evaluated by a
privacy policy.  This policy governs access to information available
from the attestation target.  

Formally, let $T_i$ be a security type that may be associated with
appraisers, attestation managers, and evidence.  Denote $p$ as
belonging to type $T_i$ by $p:T_i$.  Given an appraiser, $a:T_a$,
making a request and target, $t:T_t$, with measurable evidence
$e:T_e$, $a$ is authorized to access $e$ if the $t$'s privacy policy
allows domains of type $T_a$ access to evidence of type $T_e$. A
privacy policy for target type $T_t$ is expressed as a relation,
$\mathsf{PRIV}$:

\[\mathsf{PRIV}_{T_t}(T_a,T_e)\]

In the protocols above, \Verb+priv+ corresponds to $\mathsf{PRIV}$ and
checking privacy policy \Verb+(priv r a)+ is determining the truth
value of $\mathsf{PRIV}_{T_t}(T_a,T_e)$.  This semantics is derived
from SELinux policy developed by~\citet{Hicks:07:A-logical-speci}.




\begin{assertion}[$\mathsf{PRIV}$ is not transitive]
  $\mathsf{PRIV}_{T_t}(T_a,T_e)\wedge\mathsf{PRIV}_{T_a}(T_b,T_e)$ does
  not imply $\mathsf{PRIV}_{T_t}(T_b,T_e)$.  Informally, if $t$ trusts
  $a$ with evidence of type $T_e$ and $a$ trusts $b$ with evidence of
  type $T_e$, it does necessarily follow that $t$ trusts $b$ with
  evidence of type $T_e$.
\end{assertion}

\begin{assertion}[$\mathsf{PRIV}$ is not symmetric]
  $\mathsf{PRIV}_{T_t}(T_a,T_e)$ does not imply that
  $\mathsf{PRIV}_{T_a}(T_t,T_e)$.  In formally, if $t$ trusts $a$ with
  evidence of type $T_e$, it does not necessarily follow that $a$
  trusts $t$ with evidence of type $T_e$.
\end{assertion}

\begin{assertion}[$\mathsf{PRIV}$ is reflexive]
  $\forall t,e\cdot\mathsf{PRIV}_{T_t}(T_t,T_e)$. This is not a
  particularly useful property in that it only states $T_t$ trusts
  itself with its own evidence.
\end{assertion}

\appendix

\section*{Glossary}

\begin{description}
\item[Primary Evidence] Evidence describing the measured application.
\item[Meta-Evidence] Evidence describing properties of other evidence.
\item[Negotiation Protocol] Sequence of events used to determine what
  protocol(s) to run and when to run it/them.
\item[Attestation Protocol] Sequence of events used to gather and
  prepare evidence by invoking attestation service providers.
\item[Attestation Protocol Block (APB)] See Attestation Protocol
  Instance (API).
\end{description}

\nocite{Coker::Principles-of-R,Loscocco:98:The-Inevitabili}

\section*{Notes}

\begin{itemize}
  \squash
\item relative order of measurements (temporal ordering)
\item order measures are constructed relative to request (freshness)
\item trust status of keys
\item relevant FLASK policies that restrict communications (platform MAC)
\item vantage point from which evidence was created
\end{itemize}

\bibliography{bundling}

\end{document}
